\chapter{Validation}
\label{validate}
\index{validation}
\section{Overview}

Papers will automatically be validated when imported into the system
(including papers already present in the database), and when exported
from the system.
Validation can also be performed separately by the following command:
\begin{code}
# publish validate
\end{code}
This validates all papers found in the database (\emp{papers.pub}).

One may also validate papers in other files, which must then be stored
in the \emp{pub} format:
\begin{code}
# publish validate inputfile.pub
\end{code}

If a paper is valid, it is saved back to the database. Otherwise, it
is marked as invalid and saved to a separate file storing invalid
papers.

\section{Checks Performed}
\index{validation check}

\package{} checks each paper for a number of common errors:
\begin{itemize}
\item
  Status (status specified)
\item
  Attributes (no required attributes missing)
\item
  Venues (matching against known venues)
\item
  Titles (formatting, capitalization)
\item
  Page range (formatting)
\item
  Typos (some common typos)
\end{itemize}
We discuss each of these checks below.

\subsection{Status}
\index{status}

All papers must be marked with a status. If the status is missing, it
will automatically be set to \emp{published}. The following values
should typically be used:
\begin{itemize}
  \item
    \emp{inpreparation}
  \item
    \emp{submitted}
  \item
    \emp{accepted}
  \item
    \emp{published}
  \item
    \emp{withdrawn}
  \item
    \emp{rejected}
\end{itemize}

\subsection{Attributes}
\index{categories}

For each category, a number of attributes are required. For all
categories, the attributes \emp{title} and \emp{author} are always
required. For a list of which attributes are required for which
category, refer to Appendix~\ref{pubformat}.

\subsection{Venues}
\index{venue}

The venue (journal, conference, booktitle, etc.) for each paper is
validated against a database of known venues. If an exact match is not
found, the system tries to find a close match and suggests a
correction.

If not found, or if the suggested choice is incorrect, the user may
choose to accept the venue name as is. It will then be added to a
local file of known venues (\emp{venues.list}), which will be used for
subsequent validations.

\subsection{Titles}
\index{title}

Paper titles are validated for correct capitalization. The system
keeps a list of common words (``and'', ``or'', ``for'' etc.) that
should not be capitalized, and also a list of words that should be
capitalized. The default words for capitalization are set in the
\emp{lowercase} and \emp{uppercase} dictionaries. The default
configuration is found in
\emp{publish/config/defaults.py} and can be extended in a user's
\emp{publish\_config.py} file.

\subsection{Page Range}
\index{page range}

Page ranges must be formatted \emp{x--y}. Illegally formatted page
ranges are automatically detected and fixed.

\subsection{Typos}
\index{typos}

A number of typos are also detected. The default typos are
found in \emp{publish/config/defaults.py} and can be extended in a user's
\emp{publish\_config.py} file.
