\chapter{Importing}
\label{import}
\index{importing}

\section{Overview}

Papers may be imported into the system using the \emp{import} command
(here for a file in BibTeX format):
\begin{code}
# publish import inputfile.bib
\end{code}

Papers imported to the system will either end up in the database-file
(\emp{papers.pub}) or in a file with invalid papers that didn't pass
the first round of validation. The file with the invalid papers is
saved with date and time so that it can easily be located if needed.
During import, the papers will be merged together with the papers
already present in the database with particular attention to removing
duplicate entries.  The database is always saved in the \texttt{pub}
format.

The following steps are carried out during import (see also
Figure~\ref{fig:import}):
\begin{enumerate}
\item
  Read \emp{papers.pub}
\item
  Validate \emp{papers.pub}, see Chapter \ref{validate}
\item
  Read the file which is being imported
\item
  Validate the imported file, see Chapter \ref{validate}
\item
  Merge \emp{papers.pub} with the imported file
\item
  Make a backup-copy of the database named \emp{papers.pub.bak}
\item
  Save merge papers to \emp{papers.pub}
\end{enumerate}

\section{Supported File Formats}

In Table~\ref{tab:suffix}, the mapping between filename suffix and
file format is shown. These are the formats currently supported by
\package{}:

\begin{table}[htbp]
  \begin{center}
    \begin{tabular}{|l|l|}
      \hline
      \textbf{Filename Suffix} & \textbf{File Format}\\
      \hline
      \hline
      \emp{.pdf} & PDF\\
      \emp{.bib} & BibTeX\\
      \emp{.bibtex} & BibTeX\\
      \emp{.tex} & \LaTeX\\
      \emp{.pub} & \emp{pub}\\
      \hline
    \end{tabular}
    \caption{Mapping from filename suffix to format.}
    \label{tab:suffix}
  \end{center}
\end{table}

\section{Handling Invalid Papers}

All invalid papers will automatically be saved to a file postfixed by
the current date and time
(\emp{invalid\_papers-\%Y\%m\%d-\%H:\%m:\%S.pub}), such as for example
\emp{invalid\_papers-20081128-21:11:47.pub}.

The following steps must typically be carried out when importing papers
from a file containing one or more invalid papers:
\begin{enumerate}
  \item
    Import file
  \item
    Open file with invalid papers, edit and correct the inaccuracies
  \item
    Import the file with the invalid papers
  \item
    Repeat as needed
\end{enumerate}

Alternatively, one may also edit and validate (see next chapter) the
file containing the invalid papers before they are imported into the
database.

It is possible to import a file using the option \emp{autofix}:
\begin{code}
# publish import autofix=yes inputfile.bib
\end{code}

When autofix is used, the system automatically answers all questions
that arise during validation and merging. This may be useful (but
somewhat dangerous) when importing a large number of papers into the
database.

To see which inaccuracies that may occur during validation, see
Chapter~\ref{validate} (validation) below.

\section{Handling Duplicate Papers}

If two authors have entered the same paper in different (but similar)
ways (for example by specifying the author list in different order) or
if a conference proceeding and journal paper have identical names, a
duplicate is reported during merging. The system detects this by
computing the distance between paper venues (journal names) and paper
titles. If this distance between two papers is less than a certain
matching distance\footnote{The matching distance
\emp{matching\_distance\_strong} can be edited in
\emp{publish/config/general.py}.} then the two papers are considered
as duplicates.

If a duplicate occurs, the user is faced with the following question:
Attribute "\emph{name of attribute}" differs, what should I do?
\begin{enumerate}
\item
  Keep both papers (marking them as allowed duplicates)
\item
  Ignore papers (marking them as invalid)
\item
  Keep first paper (\emph{name of first paper}) and ignore second paper (\emph{name of second paper})
\item
  Keep second paper (\emph{name of second paper}) and ignore first paper (\emph{name of first paper})
\item
  Use attribute from first paper (\emph{value of attribute})
\item
  Use attribute from second paper (\emph{value of attribute})
\item
  Print diff 
\end{enumerate}

If the user chooses to keep both papers, they are both marked as
allowed duplicates, and will thereafter not be up for questioning
again.  If the user instead chooses to ignore the papers, they will
both be marked as invalid, and end up in the file with invalid papers.
Number three to six in the list are self explanatory. If the user
chooses to print a diff, both papers will be shown at the same time
in the terminal window, and the differences will be marked, so that
the differences are easily spotted for the user.

\section{Overriding Attributes}

When importing a file with papers, it can be useful to specify certain
attributes, for example by specifying \texttt{department=sc} (to
specify publications published by the scientific computing department)
and/or \texttt{year=2005} (to specify that all papers were published
in 2005). This means that all papers imported at this time will get
those attributes and later, if one wants to extract only those papers
written in 2005, they are easily found. (See Chapter~\ref{export}
(Exporting) for more information about paper extraction/filtering.)
The following command illustrates the above example:

\begin{code}
# publish import department=sc year=2005 inputfile.bib
\end{code}

Overriding attributes is mostly useful when the category is not easily
deduced from the BibTeX entry-type. As shown above in
Table~\ref{tab:entrytypes}, the category is not always evident for each
BibTeX entry-type. \package{} tries to make an intelligent choice, but
this is not always possible.

For example, if a paper has the BibTeX entry-type \emp{book}, then
\package{} looks for the attribute \emp{editor}. If there is such an
attribute, the \emp{book} will end up in the category \emp{edited},
and if not, the \emp{book} will end up in \emp{books}. Similarly,
papers with entry-type \emp{misc} will end up in the category
\emp{courses} if the attribute \emp{code} is found, in \emp{talks} if
the attribute \emp{meeting} is found, or else in \emp{misc}.

However, \package{} is currently not able to deduce whether a paper
with entry-types \emp{proceedings} or \emp{conference} should be
categorized into \emp{pro\-ceedings} or \emp{refproceedings}.

This must then be specified explicitly when importing papers:
\begin{code}
# publish import category=refproceedings inputfile.bib
\end{code}
If not otherwise specified, \package{} places all papers with entry-type
\emp{proceeding} (or \emp{conference}) into the category
\emp{proceedings}.
